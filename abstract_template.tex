\PassOptionsToPackage{unicode=true}{hyperref} % options for packages loaded elsewhere
\PassOptionsToPackage{hyphens}{url}
\PassOptionsToPackage{dvipsnames,svgnames*,x11names*}{xcolor}
%
\documentclass[12pt,]{article}

\usepackage{placeins}
\usepackage{float}

\usepackage[french]{babel}
\usepackage{titlesec}
% Adds page clear before each new section
\usepackage{titletoc}
\usepackage[]{utopia}
\usepackage{amssymb,amsmath}
\usepackage{ifxetex,ifluatex}
\usepackage{fixltx2e} % provides \textsubscript
\ifnum 0\ifxetex 1\fi\ifluatex 1\fi=0 % if pdftex
  \usepackage[T1]{fontenc}
  \usepackage[utf8]{inputenc}
  \usepackage{csquotes}
  \usepackage{textcomp} % provides euro and other symbols
\else % if luatex or xelatex
  \usepackage{unicode-math}
  \defaultfontfeatures{Ligatures=TeX,Scale=MatchLowercase}
\fi
% use upquote if available, for straight quotes in verbatim environments
\IfFileExists{upquote.sty}{\usepackage{upquote}}{}
% use microtype if available
\IfFileExists{microtype.sty}{%
\usepackage[]{microtype}
\UseMicrotypeSet[protrusion]{basicmath} % disable protrusion for tt fonts
}{}
\IfFileExists{parskip.sty}{%
\usepackage{parskip}
}{% else
\setlength{\parindent}{0pt}
\setlength{\parskip}{6pt plus 2pt minus 1pt}
}
\usepackage{xcolor}
\usepackage{hyperref}
\hypersetup{
            pdftitle={Internship Report Abstract},
            pdfauthor={Corentin Gay GISTRE 2018; Intenship Mentor : Anthony Leonardo Gracio},
            colorlinks=true,
            linkcolor=Maroon,
            citecolor=Blue,
            urlcolor=blue,
            breaklinks=true}
\urlstyle{same}  % don't use monospace font for urls
\usepackage{graphicx,grffile}
\makeatletter
\def\maxwidth{\ifdim\Gin@nat@width>\linewidth\linewidth\else\Gin@nat@width\fi}
\def\maxheight{\ifdim\Gin@nat@height>\textheight\textheight\else\Gin@nat@height\fi}
\makeatother
% Scale images if necessary, so that they will not overflow the page
% margins by default, and it is still possible to overwrite the defaults
% using explicit options in \includegraphics[width, height, ...]{}
\setkeys{Gin}{width=\maxwidth,height=\maxheight,keepaspectratio}
\FloatBarrier
\setlength{\emergencystretch}{3em}  % prevent overfull lines
\providecommand{\tightlist}{%
  \setlength{\itemsep}{0pt}\setlength{\parskip}{0pt}}
\setcounter{secnumdepth}{5}
% Redefines (sub)paragraphs to behave more like sections
\ifx\paragraph\undefined\else
\let\oldparagraph\paragraph
\renewcommand{\paragraph}[1]{\oldparagraph{#1}\mbox{}}
\fi
\ifx\subparagraph\undefined\else
\let\oldsubparagraph\subparagraph
\renewcommand{\subparagraph}[1]{\oldsubparagraph{#1}\mbox{}}
\fi

% set default figure placement to htbp
\makeatletter
\def\fps@figure{H}
\makeatother


\clearpage
\title{Internship Report Abstract}
\author{Corentin Gay GISTRE 2018 \and Intenship Mentor : Anthony Leonardo Gracio}
\date{From 19/02/2018 to 24/08/2018 (6 months)}

\begin{document}
\addtocontents{toc}{\protect\thispagestyle{empty}}

\maketitle
\begin{figure}
\centering
	\includegraphics{adacore.jpg}
	\includegraphics[height=7cm]{epita.png}
\end{figure}
\FloatBarrier

\thispagestyle{empty}
\clearpage

{
\hypersetup{linkcolor=}
\setcounter{tocdepth}{3}
\pagenumbering{gobble}
\addtocontents{toc}{\protect\thispagestyle{empty}}
\setcounter{tocdepth}{3}
}

\pagenumbering{arabic}

\hypertarget{presentation}{%
\section{Presentation}\label{presentation}}

AdaCore is a company specializing in compiler technology. They develop
and maintain an Ada compiler named GNAT. They target embedded targets
with or without operating systems. They offer technical expertise to
their clients, allowing them to easily inquire about existing features
or report technical issues.

The team I worked in is called the IDE team. Its goal is to maintain and
develop the Ada IDE of AdaCore called GPS (GNAT Programming Studio).
This IDE integrates multiple tools specific to Ada, allowing developpers
to be more efficient in their work.

My mentor is Anthony Leonardo Gracio. He handles all the graphical
display of the IDE. He is an former EPITA student. I also interacted a
lot with Fabien Chouteau (also a former EPITA student) that is part of
the \enquote{bare-board} team. He handles work related to embedded
devices and makes sure that the AdaCore products are working on the
different platforms that the clients use.

My work was focused on improving the bare-metal\footnote{To have an
  overview of their work, go
  \href{https://www.adacore.com/industries}{here}} support in GPS.
Bare-metal development in Ada requires more work than in C because Ada
code needs a runtime in order to compile and run. An Ada runtime
implements the functionalities of the language such as multi-tasking
support or memory allocation.

In order to target a new platform, one must recompile a runtime adapted
to the platform. This is something that is extremely repetitive. For
example, the code behind the runtimes was the same, it was just compiled
differently.

We want to avoid having to do that work each time. So, I settled on
using the CMSIS-Packs. It is a standard designed by ARM that describes
harware and software of a device family using the Cortex-M processors.
Using the information in those packs, we can generate the code that is
specific to our target. In our case, we generate two files. First, a
linker script describing how the executable is to be mapped in physical
memory. Second, an assembly file called the \enquote{startup code} that
takes care of initializing memory regions before the execution of the
user program. The final goal was to be able to start developping on a
board from GPS right away without having to look up documentation to
configure the different files.

I was expected to have some knowledge bare metal programming and how
compilers work. I was also expected to know how to present my work as we
had monthly intern presentations. I needed to be able to research
efficiently and be able to design program architectures. Finally, I
needed to be able to communicate efficiently with my colleagues as I had
to integrate the project that I had made in the work of other people.

I demonstrated my working prototype in the start of july during my
monthly presentation. I am currently working on integrating my work into
GPS.

\hypertarget{intenship-schedule}{%
\section{Intenship Schedule}\label{intenship-schedule}}

The first phase of my internship was a research phase. Thus. I did not
plan a weekly schedule. Here is the the schedule :

INSERT SCHEMA HERE

\hypertarget{general-appreciation}{%
\section{General appreciation}\label{general-appreciation}}

I really enjoyed my time at AdaCore. I had great colleagues to which I
could ask technical questions and get great answers. I could even
contribute to the code of the project the team was maintaining. When I
found bugs, I reported them and made sure the fix worked on my machine.

\hypertarget{skills-acquired-during-the-internship}{%
\section{Skills acquired during the
internship}\label{skills-acquired-during-the-internship}}

I acquired a lot of knowledge regarding compiler technologies. I also
learned how to better code in Ada as I only had a basic understanding of
the language features. I imporved my skills at reading unfamiliar code.
When I discovered a bug in the tools I was using, I peeked in the code
in order to understand the different systems working together and where
the bug was coming from. I learned a lot doing that and I found it
really enjoyable.

\hypertarget{conclusion}{%
\section{Conclusion}\label{conclusion}}

I had a very positive experience at AdaCore. I learned a lot and they
work with great clients on amazing projects\footnote{To have an overview
  of their work, go \href{https://www.adacore.com/industries}{here}}.
During my internship, I candidated for an open position they had. I
received an offer that I accepted and I plan on starting by the end of
September



\nocite{pizza}

\end{document}
