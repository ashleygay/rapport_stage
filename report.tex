\PassOptionsToPackage{unicode=true}{hyperref} % options for packages loaded elsewhere
\PassOptionsToPackage{hyphens}{url}
%
\documentclass[11pt,]{article}
\usepackage{graphicx}
\usepackage{titlesec}
% Adds page clear before each new section
\newcommand{\sectionbreak}{\clearpage}
\usepackage{titletoc}
\usepackage[]{utopia}
\usepackage{amssymb,amsmath}
\usepackage{ifxetex,ifluatex}
\usepackage{fixltx2e} % provides \textsubscript
\ifnum 0\ifxetex 1\fi\ifluatex 1\fi=0 % if pdftex
  \usepackage[T1]{fontenc}
  \usepackage[utf8]{inputenc}
  \usepackage{textcomp} % provides euro and other symbols
\else % if luatex or xelatex
  \usepackage{unicode-math}
  \defaultfontfeatures{Ligatures=TeX,Scale=MatchLowercase}
\fi
% use upquote if available, for straight quotes in verbatim environments
\IfFileExists{upquote.sty}{\usepackage{upquote}}{}
% use microtype if available
\IfFileExists{microtype.sty}{%
\usepackage[]{microtype}
\UseMicrotypeSet[protrusion]{basicmath} % disable protrusion for tt fonts
}{}
\IfFileExists{parskip.sty}{%
\usepackage{parskip}
}{% else
\setlength{\parindent}{0pt}
\setlength{\parskip}{6pt plus 2pt minus 1pt}
}
\usepackage{hyperref}
\hypersetup{
            pdftitle={Rapport de stage de fin d'etudes},
            pdfauthor={X fevrier au 24 aout},
            pdfborder={0 0 0},
            breaklinks=true}
\urlstyle{same}  % don't use monospace font for urls
\usepackage{color}
\usepackage{fancyvrb}
\newcommand{\VerbBar}{|}
\newcommand{\VERB}{\Verb[commandchars=\\\{\}]}
\DefineVerbatimEnvironment{Highlighting}{Verbatim}{commandchars=\\\{\}}
% Add ',fontsize=\small' for more characters per line
\usepackage{framed}
\definecolor{shadecolor}{RGB}{48,48,48}
\newenvironment{Shaded}{\begin{snugshade}}{\end{snugshade}}
\newcommand{\KeywordTok}[1]{\textcolor[rgb]{0.94,0.87,0.69}{#1}}
\newcommand{\DataTypeTok}[1]{\textcolor[rgb]{0.87,0.87,0.75}{#1}}
\newcommand{\DecValTok}[1]{\textcolor[rgb]{0.86,0.86,0.80}{#1}}
\newcommand{\BaseNTok}[1]{\textcolor[rgb]{0.86,0.64,0.64}{#1}}
\newcommand{\FloatTok}[1]{\textcolor[rgb]{0.75,0.75,0.82}{#1}}
\newcommand{\ConstantTok}[1]{\textcolor[rgb]{0.86,0.64,0.64}{\textbf{#1}}}
\newcommand{\CharTok}[1]{\textcolor[rgb]{0.86,0.64,0.64}{#1}}
\newcommand{\SpecialCharTok}[1]{\textcolor[rgb]{0.86,0.64,0.64}{#1}}
\newcommand{\StringTok}[1]{\textcolor[rgb]{0.80,0.58,0.58}{#1}}
\newcommand{\VerbatimStringTok}[1]{\textcolor[rgb]{0.80,0.58,0.58}{#1}}
\newcommand{\SpecialStringTok}[1]{\textcolor[rgb]{0.80,0.58,0.58}{#1}}
\newcommand{\ImportTok}[1]{\textcolor[rgb]{0.80,0.80,0.80}{#1}}
\newcommand{\CommentTok}[1]{\textcolor[rgb]{0.50,0.62,0.50}{#1}}
\newcommand{\DocumentationTok}[1]{\textcolor[rgb]{0.50,0.62,0.50}{#1}}
\newcommand{\AnnotationTok}[1]{\textcolor[rgb]{0.50,0.62,0.50}{\textbf{#1}}}
\newcommand{\CommentVarTok}[1]{\textcolor[rgb]{0.50,0.62,0.50}{\textbf{#1}}}
\newcommand{\OtherTok}[1]{\textcolor[rgb]{0.94,0.94,0.56}{#1}}
\newcommand{\FunctionTok}[1]{\textcolor[rgb]{0.94,0.94,0.56}{#1}}
\newcommand{\VariableTok}[1]{\textcolor[rgb]{0.80,0.80,0.80}{#1}}
\newcommand{\ControlFlowTok}[1]{\textcolor[rgb]{0.94,0.87,0.69}{#1}}
\newcommand{\OperatorTok}[1]{\textcolor[rgb]{0.94,0.94,0.82}{#1}}
\newcommand{\BuiltInTok}[1]{\textcolor[rgb]{0.80,0.80,0.80}{#1}}
\newcommand{\ExtensionTok}[1]{\textcolor[rgb]{0.80,0.80,0.80}{#1}}
\newcommand{\PreprocessorTok}[1]{\textcolor[rgb]{1.00,0.81,0.69}{\textbf{#1}}}
\newcommand{\AttributeTok}[1]{\textcolor[rgb]{0.80,0.80,0.80}{#1}}
\newcommand{\RegionMarkerTok}[1]{\textcolor[rgb]{0.80,0.80,0.80}{#1}}
\newcommand{\InformationTok}[1]{\textcolor[rgb]{0.50,0.62,0.50}{\textbf{#1}}}
\newcommand{\WarningTok}[1]{\textcolor[rgb]{0.50,0.62,0.50}{\textbf{#1}}}
\newcommand{\AlertTok}[1]{\textcolor[rgb]{1.00,0.81,0.69}{#1}}
\newcommand{\ErrorTok}[1]{\textcolor[rgb]{0.76,0.75,0.62}{#1}}
\newcommand{\NormalTok}[1]{\textcolor[rgb]{0.80,0.80,0.80}{#1}}
\usepackage{graphicx,grffile}
\makeatletter
\def\maxwidth{\ifdim\Gin@nat@width>\linewidth\linewidth\else\Gin@nat@width\fi}
\def\maxheight{\ifdim\Gin@nat@height>\textheight\textheight\else\Gin@nat@height\fi}
\makeatother
% Scale images if necessary, so that they will not overflow the page
% margins by default, and it is still possible to overwrite the defaults
% using explicit options in \includegraphics[width, height, ...]{}
\setkeys{Gin}{width=\maxwidth,height=\maxheight,keepaspectratio}
\setlength{\emergencystretch}{3em}  % prevent overfull lines
\providecommand{\tightlist}{%
  \setlength{\itemsep}{0pt}\setlength{\parskip}{0pt}}
\setcounter{secnumdepth}{5}
% Redefines (sub)paragraphs to behave more like sections
\ifx\paragraph\undefined\else
\let\oldparagraph\paragraph
\renewcommand{\paragraph}[1]{\oldparagraph{#1}\mbox{}}
\fi
\ifx\subparagraph\undefined\else
\let\oldsubparagraph\subparagraph
\renewcommand{\subparagraph}[1]{\oldsubparagraph{#1}\mbox{}}
\fi

% set default figure placement to htbp
\makeatletter
\def\fps@figure{htbp}
\makeatother


\clearpage
%TODO: fix images for on the title page
\title{Rapport de stage de fin d'etudes}
\author{X fevrier au 24 aout}
\date{Corentin Gay}

\begin{document}
\maketitle
\begin{figure}
\centering
	\includegraphics{adacore.jpg}
	\includegraphics[height=7cm]{epita.png}
\end{figure}

\thispagestyle{empty}
\clearpage

{
\setcounter{tocdepth}{3}
\tableofcontents
\thispagestyle{empty}
\clearpage
}
\section{Resume}\label{resume}

J'ai decouvert Ada lors des cours a EPITA donnes par Raphael Amiard.
J'ai trouve le langage tres interessant car les concepts (notamment
l'oriente objet) sont assez differents de leurs equivalents en C++. Lors
de ce cours, nous avions essaye de faire fonctionner un emulateur
Gameboy code en C++ avec un programme Ada qui devait se charger de
l'affichage, tout cela tournant sur une carte STM32F729. Les difficultes
technique de ce projet m'ont passione et j'ai

parler de faire des trucs embedded mais d'avoir a les integrer dans un
IDE ?? possibilite d'en apprendre plus sur le fonctionnement des
runtimes et de leur role dans le developpement embarque en Ada

Plusieurs facteurs ont affecte mon choix pour ce stage. Le premier est
l'opportunite d'integrer mon travail dans un outil pre-existant. Ce
point me semble important car il permet de comprendre le contexte
technique dans lequel sera utilise mon travail. Le second point est le
sujet qui me permet de toucher a plusieurs technologies tout en restant
dans mon domaine de predilection, le developpement embarque. Par
exemple, en apprendre plus sur les differents processeurs ARM et leurs
assembleurs differents me paraissait etre une bonne experience a avoir.

En Ada, il existe un concept de runtime, cette runtime est le logiciel
qui va permettre d'utiliser certaines fonctionnalites du langage (par
exemple : un allocateur memoire).

Plusieurs types de runtime, 2 principaux types:

\begin{itemize}
\tightlist
\item
  runtimes ravenscar : possedent presque les features d'un RTOS
\item
  runtimes ZFP : minimum pour faire tourner du code Ada
\end{itemize}

Par exemple, la ZFP fournit un allocateur memoire utilisant une pile,
avec une fonction \texttt{free} qui ne fait rien. Etant donne cet
allocateur memoire il est impossible d'avoir une propagation d'exception
efficace car on ne peut pas recupere la memoire allouee lorsqu'une
exception est levee.

Dans le contexte de ce stage nous ne parlerons que de runtimes ZFP.

C'est quoi le linker script et le startup code ????

Dans ce cadre la, lorsque je veux developper pour une nouvelle cible en
compilation croisee, il faut recompiler une runtime pour en generer une
adaptee a la cible. Pour cela il faut modifier le
\texttt{linker\ script} pour representer le \texttt{mapping\ memoire} et
le \texttt{startup\ code} afin d'utiliser l'assembleur de la cible.

\begin{Shaded}
\begin{Highlighting}[]
\KeywordTok{with}\NormalTok{ Ada.Text_IO;}

\KeywordTok{procedure}\NormalTok{ Test (Entier : }\DataTypeTok{Integer}\NormalTok{);}
\NormalTok{Test : }\DataTypeTok{Integer}\NormalTok{ := }\DecValTok{8}\NormalTok{;}
\NormalTok{Test : }\DataTypeTok{Integer}\NormalTok{ := }\DecValTok{8}\NormalTok{;}
\NormalTok{Test : }\DataTypeTok{Integer}\NormalTok{ := }\DecValTok{8}\NormalTok{;}
\NormalTok{Test : }\DataTypeTok{Integer}\NormalTok{ := }\DecValTok{8}\NormalTok{;}
\end{Highlighting}
\end{Shaded}

On peut diviser le contenu d'une runtime en trois grandes parties :

\begin{itemize}
\tightlist
\item
  cpu specifiques : architecture and float handling
\item
  device specific : memory mapping
\item
  board specific : additionnal mem mapping + device mapping
\end{itemize}

Dans le cas de ZFP, le code de la runtime elle-meme est identique d'un
materiel a l'autre, c'est le contenu du \texttt{linker\ script} et du
\texttt{startup\ code} qui va changer.

Heureusement, ARM a creer un standard qui permet de decrire le materiel
d'une famille de boards et de devices. Ce sont les CMSIS-Packs

A partir de ces packs, on a toute les informations necessaires pour
generer un linker script et le startup code pour une board donnee.

\section{Introduction}\label{introduction}

Contexte et complexite du stage

\subsection{Rappel du sujet de stage}\label{rappel-du-sujet-de-stage}

Mon sujet de stage s'intitulait `Improve baremetal support in GPS' et
comportait plusieurs axes. Par exemple : ameliorer la stack view ou bien
developper une register view ou encore, `investiguer' comment rendre la
creation d'un projet plus simple et plus generique.

J'ai choisi de m'attaquer a ce dernier. L'idee etait de fournir les
outils a l'utilisateur fin qu'il puisse choisir sa cible de
developpement et que GPS genere les fichiers necessaires afin de pouvoir
commencer a developper apres la creation du projet.

Cependant pour pouvoir executer du code Ada sur une cible donnee, il
faut avoir un logiciel appelle une \texttt{runtime}. Ce logiciel fournit
des fonctionnalitees du langage qui sont utilisee par le programme comme
le support multi-taches, la propagation des exceptions ou un allocateur
memoire.

Etant donne que le code etant dans la runtime doit tourner sur la cible,
il faut adapter la runtime a chaque cible. Actuellement, c'est une etape
qui est fait manuellement. Il faut egalement savoir qu'il y a plusieurs
types de runtime qui ne fournissent pas toutes les memes
fonctionnatlites. Dans le cas de mon stage je me suis attaque a la
question des \texttt{runtimes} dites \texttt{ZFP} pour
\texttt{zero\ footprint}. Ce type de runtime est le minimum pour pouvoir
faire tourner du code Ada. Par exemple, elle n'a pas de propagation
d'exception, pas de support multi-taches et ne possede qu'un allocateur
memoire naif.

Dans le cas decrit ci-dessus, les modifications a effectuer dans le code
de la runtime elle-meme sont nulles. Cependant, il faut tout de meme
modifier le \texttt{crt0} qui permet de preparer les differentes
memoires et qui appelle la fonction \texttt{main}. Il faut egalement
modifier le \texttt{linker\ script}, le fichier responsable de la
cartographie memoire, afin de specifier les zones memoires et quelles
portions du code y mettre.

Ces modifications dependent de la cible et necessite actuellement de
lire la documentation afin de recuperer les infos utiles. Dans le cas
d'une cible possedant un processeur cortex-m, ARM a creer un standard
qui permet de decrire le materiel d'une \texttt{board} ou d'un
\texttt{device} et \texttt{packageant} ces infos dans une archive zip.
On appelle ces archives des CMSIS-Packs.

La solution est donc d'utiliser ces packs pour automatiser le processus
de modification du \texttt{startup\ code} et du \texttt{linker\ script}.
Cette automatisation permet a l'utilisateur de choisir sa \texttt{board}
de developpement lors de la creation d'un projet

\subsection{Presentation de
l'entreprise}\label{presentation-de-lentreprise}

En 1992, l'universite de New York conclut un contrat avec
l'\texttt{US\ Air\ Force} afin de creer un compilateur libre et standard
afin d'aider a la diffusion du nouveau standard Ada, Ada 9X (qui
deviendra Ada 95). Suite a ce projet, la societe Ada Core Technologies
est cree a New York et la societee soeur ACT-Europe est cree deux annees
plus tard. Ce n'est qu'en 2012 que les deux societes sont unifiees.

AdaCore fournit un compilateur Ada appelle GNAT en plusieurs versions
avec des licenses differentes. La version chaque version des
fonctionnalite differentes, par exemple la version \texttt{Community} ne
supporte que la derniere version du standard Ada, Ada 2012, alors que la
version \texttt{Assurance} destinee aux projets de certifications ou a
des projets de longues durees supporte jusqu'a Ada 83. De plus, avec la
version \texttt{Community}, tout le code ecrit est soumis a la license
GPL tandis qu'avec la version commerciale, une exception est presente
dans la license permettant de ne pas etre soumis a la GPL.

Pour aller avec le compilateur, AdaCore peut egalement aider les clients
avec des projets de certifications. En effet, une partie des outils
fournis par AdaCore, comme GNATcoverage, est qualifie pour le
developpement d'outil en DO-178B en DAL A. C'est a dire le niveau de
criticite le plus eleve dans l'industrie avionique. GNATcoverage aide a
l'analyse de couverture de code ce qui permet de garantie qu'il n'y a
pas de code qui n'est jamais execute.

AdaCore a beaucoup de clients dans des domaines ou la presence d'erreurs
n'est pas acceptable comme le domaine de l'avionique ou de la defense.
Voici quelques projets que des clients d'AdaCore ont realises :

\begin{itemize}
\item
  MDA, une division de Maxar Technologies, va utiliser Ada ainsi que le
  produit GNAT Pro Assurance afin de realiser le logiciel en charge de
  la communication espace-terre a bord de l'ISS.
\item
  Real Heart AB est une entreprise suedoise qui travaille sur un coeur
  totalement artificiel. Afin de garantir le bon fonctionnement du
  logiciel qui pilote le moteur de la pompe du coeur artificiel, elle a
  choisi d'utiliser Ada ainsi que le compilateur GNAT Pro fourni par
  AdaCore.
\end{itemize}

AdaCore travaille sur un compilateur Ada, GNAT. Ils fournissent des
compilateurs croises qui visent une plate-forme specifique a leurs
clients. Ils offrent egalement du support pro sur tous leurs produits.
Parler des differentes runtimes ? Domaines : defense, aeorospatiale,
securite (voir site)

Mon stage se situe dans la perspective d'ameliorer l'experience des
utilisateurs de GPS dans le domaine du \texttt{bare\ board}.

Thematiques du stage : bare board, IDE et CMSIS-Packs Actuellement pas
d'integration des CMSIS-Packs dans GPS. Contrairement a Eclipse qui
supporte parfaitement les pack.

Mon stage se deroule dans l'equipe IDE. Cette equipe s'occupe de la
maintenace de l'IDE GPS. Cet IDE utilise les references croisees afin de
fournir une meilleure experience de developpement a l'utilisateur. Mon
stage s'insere donc parfaitement dans les thematiques de cette equipe.

Ce support est specifique au langage Ada, le concepte de runtime est
plus ou moins unique a ce langage.

Rapport a epita: j'ai fait du bareboard et de l'Ada. J'avais deja essaye
de faire un projet mixant Ada et C++, mais ce ne s'etait pas fini comme
prevu. Pas de compilateur ds la toolchain d'AdaCore. Motivation: ca
touchait a du bare metal, mais il fallait quand meme integrer ca dans un
ide `classique' (en Ada lol)

\section{Aspects organisationnels}\label{aspects-organisationnels}

\subsection{Decoupage du stage}\label{decoupage-du-stage}

\subsection{Diagramme de Gantt, Kanban
??}\label{diagramme-de-gantt-kanban}

\subsection{Points de controle}\label{points-de-controle}

\subsection{Situations de Crise ?????
Kesako}\label{situations-de-crise-kesako}

\section{Aspects techniques}\label{aspects-techniques}

Liste:

\begin{itemize}
\tightlist
\item
  generation du startup code
\item
  generation du linker script
\item
  base de donnees representant les packs
\item
  integration dans GPS
\end{itemize}

\subsection{Objectifs}\label{objectifs}

\subsubsection{Alternatives}\label{alternatives}

\subsection{Cadre du stage dans
l'entreprise}\label{cadre-du-stage-dans-lentreprise}

\subsection{Propositions retenues ou
pas}\label{propositions-retenues-ou-pas}

on ne genere pas des runtimes on prend celles de bb\_runtimes
probablement par raison politique, le code de la runtime n'est pas
ouvert au public \#\# Difficultes eventuelles \#\# Resultats obtenus
avancement

\section{Bilan}\label{bilan}

\begin{itemize}
\tightlist
\item
  pretty good
\end{itemize}

\end{document}
